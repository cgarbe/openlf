%
\setlength{\parindent}{0em} 
%
\colorlet{shadecolor}{gray!25}
%
\section{Introduction}
%
\vspace{4mm}
%
This toolbox allows to visualize vertical, horizontal and cross structured light field.
%
The light fields are displayed either as animation or as EPIs at the selected position in the center viewer.
%
In this viewer it is possible to interactively change the horopter real time (EPI mode) and apply it to the entire light field to watch the related animation afterwards.
%
Additionally it is possible to save all shifted images to generate an gif animation of the underlying light field at a given Horopter.
%
This all is a very supporting in terms of light field analysis, especially the zooming function in the animation window.\\
%
Aside this also the openCV camera calibration for single cameras is implemented.
%
To use the calibration, a modified version of the xml-config file, provided by openCV needs to be loaded to unlock the calibration features in the control terminal.\\
%
At last, the structure tensor orientation estimation with occlusion handling is implemented and can be executed in different modes such as single or accumulate mode.\\
%
A detailed explanation of each toolbox part is given in the following sections.
%
For each part different examples are provided in the data-subfolder.\\
%
\begin{figure}[htb]
	\begin{minipage}[h]{\textwidth}
		\centering
		\begin{overpic}[width=0.9\linewidth]{Images/LF_Viewer/Frontend_LFViewer.png}
		\end{overpic}
	\end{minipage}
	\label{figure_estimationRange}
	\caption{Illustrates the toolbox front end without any loaded ini-file. In the initial state no segments are unlocked. Dependent on the loaded configuration file different segments with its buttons will unlock.}
\end{figure}

% 
\section{Light-Field Viewer}
%
To visualize light fields all related images need to be located in a folder which is defined in the ini-file, see figure~\ref{figure_INI_example}. 
%
In addition also the type of the light field is a necessary information to display the data correctly.
%
The viewer supports three different kind of light fields which are horizontal, vertical or cross light-fields.
%
To load cross light fields it is important to consider that at the given location defined in the ini-file, a 'h'-folder for the horizontal and a 'v'-folder for the vertical light field direction is present in which the related images are located.
%
This is important, because in contrast to horizontal or vertical light fields, for cross light fields the software searches the image data in the two described sub-folders instead of the current location. 
%
As third parameter the horopter of the light field can be set optionally, to place the center position of the slider to an initial horopter value unequal zero.
%
The last option is to invert the images direction.
%
Image sequences having the inverted order are visualized correctly but the disparity becomes negative.
%
To avoid this without renaming the images it is possible to invert the load order.
%
\vspace{2mm}
%
\begin{figure}[htb]
	\begin{minipage}[h]{\textwidth}
		\centering
		\begin{overpic}[width=0.95\linewidth]{Images/LF_Viewer/ini-File_example.png}
		\end{overpic}
	\end{minipage}
	\caption{Shows the content of an ini-file. For the visualization only the parameter are needed. The first is the location of the images. The second the type of light field and the third one is a preset horopter which is optional.}
	\label{figure_INI_example}
\end{figure}
\\
%
After loading the light field into the viewer the buttons and input fields of the light-field viewer as well as for the structure-tensor orientation estimation become available.
%
To set the parameter for the structure tensor the values can entered directly in the input  fields or can be preset in the ini-file as shown in figure~\ref{figure_INI_example2}.
%
\begin{figure}[htb]
	\begin{minipage}[h]{\textwidth}
		\centering
		\begin{overpic}[width=0.7\linewidth]{Images/LF_Viewer/ViewerWithLoadedInI.png}
		\end{overpic}
	\end{minipage}
	\caption{Shows the front end of the LF Viewer after loading an ini-file. The buttons and input areas of the light-field viewer and the structure tensor orientation estimation section are available now.}
	\label{figure_LFViewer_frontend_withINI}
\end{figure}

%
Aside the unlocked buttons also the main window opens, see figure~\ref{figure_Window} on the left side.
% 
It contains the center view of the light field and shows all available EPI (vertical and horizontal).
% 
The shown light file is part of the Middlebury dataset and can be seen as sparsely sampled light field containing 7 views in horizontal direction captured.
%
Thus the horizontal EPI contains information from $7$ views while the vertical EPI just contains one column. 
%
When the main window gets closed it can be re-opened by clicking the monitor button in the control terminals menu bar.
%
The other monitor button, having the play symbol inside starts the light field animation for the applied horopter which opens in its own window, as shown in figure~\ref{figure_Window} on the right side. 
%
This animation window is controllable with some keys and the mouse to evaluate the light fields in more detail.\\
%
\newline
%
The available keys are:
%
\begin{table}[h]
\begin{tabular}{c|l}
ESC & Close Window \\ \hline
S & Display next image and stop \\ 
& (to leave press any other key) \\ \hline
Space & Change from vertical in horizontal direction and back.\\ \hline
Mouse wheel & Zoom in and out \\ \hline
Mouse click and hold & Move the image position (in zoomed images)  
\end{tabular}
\end{table}
%
\\
%
To see the light field at a different horopter it is important to press the button "Apply Horopter to light field".
%
After pressing the button the selected Horopter gets applied to the entire light field.
%
The animation window updates the output automatically if the computation of the new horopter is finished.   
%
While watching the animation or when saving the light field at different horopter, it is possible to cut the light field in vertical (LF-cut Y) or horizontal (LF-cut X) direction to remove the appearing black borders due to the applied horopter.
%
\begin{figure}[htb]
	\begin{minipage}[h]{0.48\textwidth}
		\centering
		\begin{overpic}[width=0.9\linewidth]{Images/LF_Viewer/MainWindow.png}
		\end{overpic}
	\end{minipage}
	\begin{minipage}[h]{0.48\textwidth}
		\centering
		\begin{overpic}[width=0.9\linewidth]{Images/LF_Viewer/Animation.png}
		\end{overpic}
	\end{minipage}
	\caption{Illustrated the toolbox front end without any loaded ini-file. In the initial state no segments are unlocked. Dependent on the loaded configuration file different segments with its buttons will unlock.}
	\label{figure_Window}
\end{figure}

%
\newpage
%
\section{Structure Tensor Orientation Estimation with Occlusion Handling}
%
For each loaded light field also the structure tensor orientation estimation with occlusion handling can be applied. 
%
Thus it is important to set the result location in the ini-file, as shown in figure~\ref{figure_INI_example2}.
%
At this location the results of the structure tensor computation is saved automatically which is important to set because there is no visualization output of structure tensor result.
%
By pressing the button "Apply ST Occlusion Handling(local)" the algorithm computes the disparity for the light field at the given horopter and saves the result.
%
When changing the horopter and applying the algorithm again by pressing the button "Apply ST Occlusion Handling(local)" the old result gets overwritten.
%
To avoid this the accumulate flag needs to be set.
%
Now the results of each local estimation are combined by taking for each pixel the disparity value having the largest coherence value.
%
To compute light fields having a large disparity range one can also define the start and the stop disparity and press "Run 'global shift'".
%
The algorithm computes now the light field in the entire range automatically and merges the views.\\
%
The implemented structure tensor uses bilateral filter instead of Gaussian filter which is important for the occlusion handling.
%
Because of this an additional intensity threshold can be set.
%
To preset the parameter, it is also possible to define the parameter in the ini-file as already shown in figure~\ref{figure_INI_example2}\\
%
\begin{figure}[htb]
	\begin{minipage}[h]{\textwidth}
		\centering
		\begin{overpic}[width=0.9\linewidth]{Images/LF_Viewer/ini-File_example2.png}
		\end{overpic}
	\end{minipage}
	\caption{Illustrates the content of the ini-file with parameters for the structure tensor section.}
	\label{figure_INI_example2}
\end{figure}

%
\newpage
%
\section{OpenCV Camera Calibration}
%
\begin{figure}[htb]
	\begin{minipage}[h]{\textwidth}
		\centering
		\begin{overpic}[width=0.7\linewidth]{Images/LF_Viewer/ViewerWithXMLFileLoaded.png}
		\end{overpic}
	\end{minipage}
	\caption{Illustrated the toolbox front end without any loaded ini-file. In the initial state no segments are unlocked. Dependent on the loaded configuration file different segments with its buttons will unlock.}
	\label{figure_LFViewer_frontend_withXML}
\end{figure}

%
To use the openCV camera calibration, a XML-file as shown in figure~\ref{figure_openCV_XML} is needed to set all needed parameter.
%
The first parameter "CalibrationImages" defines the location of the calibration images while the second parameter "ImagesToRectify" defines the location of the images used for the undistortion.
%
All other parameter are calibration related and explained in the XML-file itself.
%
The calibration result is saved in the same folder as the xml-file is located and is named as defined by the parameter "Write\_outputFileName".
%
A calibration result is shown in figure~\ref{figure_calib_result}.
%
Saved is the camera matrix and the distortion coefficients also the path to the images to undistort as already defined in the xml-file.
%
This parameter value appears at this position again to apply the undistortion with the already computed calibration to different image sequences without recomputing the calibration each time.
%
Thus only the file, as named at "Write\_outputFileName" in the xml-file, containing the calibration result needs to be reloaded.
%
Then the "openCV\_undistort" button unlocks and the undistortion can be applied to the related images.
%
The undistortion also unlocks right after the calibration is finished successfully.
%
To run both at once the button "openCV\_calib + undist" needs to be clicked where right after the calibration the images get undistorted.
%
\begin{figure}[htb]
	\begin{minipage}[h]{\textwidth}
		\centering
		\begin{overpic}[width=0.9\linewidth]{Images/LF_Viewer/openCV_XML.png}
		\end{overpic}
	\end{minipage}
	\caption{Shows the modified version of the openCV XML-file to calibrate single cameras. It is used to apply the calibration with its necessary parameter but also defines a location to apply the image undistortion right after the calibration.}
	\label{figure_openCV_XML}
\end{figure}

%
\begin{figure}[htb]
	\begin{minipage}[h]{\textwidth}
		\centering
		\begin{overpic}[width=0.9\linewidth]{Images/LF_Viewer/Calib_result.png}
		\end{overpic}
	\end{minipage}
	\caption{Shows the result of the openCV camera calibration. It is automatically saved in a file called \_calibration\_data.ini. It contains the camera matrix values and the distortion coefficients necessary to un-distort the images.}
	\label{figure_calib_result}
\end{figure}

%
